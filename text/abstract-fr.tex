\begin{otherlanguage}{french} 
Les langages scientifiques de haut niveau, tels que MATLAB et Python                                
et sa librairie NumPy, gagnent en popularité auprès des scientifiques                               
et des mathématiciens.  Ces langages offrent des fonctionalités telles                              
que le typage dynamique et des fonctions scientifiques de haut niveau                               
qui permettent un prototypage facile.  Par contre, ces fonctionalités                               
diminue la performance en exécution du code.  

Nous présentons VeloCty,                              
un compilateur statique optimisant pour MATLAB et Python comme                                      
solution au problème d'améliorer la performances des programmes écrits                              
dans ces langages.  Pour la majorité des programmes, une grande                                     
proportion du temps d'exécution est passée à exécuter une petite                                    
section du code.  De plus, ces sections peuvent souvent être compilées                              
avant l'exécution du code et on peut obtenir une amélioration en                                    
performance en optimisant seulement ces sections chaudes.  VeloCty                                  
prend en entrée des fonctions écrites en MATLAB et Python spécifiées                                
par l'utilisateur et génère une version équivalente en C++.  VeloCty                                
génère également le code d'interfaçage pour l'intégration avec MATLAB                               
et Python.  Le code généré peut ainsi être compilé comme une                                        
bibliothèque partagée qui peut être liée avec n'importe quel programme                              
écrit en MATLAB et Python.  Nous implémentons aussi des optimisations                               
pour éliminer les tests de bornes des tableaux, pour réutiliser de la                               
mémoire déjà allouée dans les opérations sur les tableaux, et pour                                  
supporter l'exécution parallèle via OpenMP.                                                         
                                                                                                    
VeloCty utilise le système de compilation Velociraptor.  Nous                                       
implémentons un générateur de code qui transforme la représentation                                 
intermédiaire de Velociraptor, VRIR, en C++ ainsi que des supports                                  
d'exécution spécifiques pour MATLAB et Python.  Nous avons également                                
implémenté un générateur de code MATLAB à VRIR à l'aide de McLab.                                   
                                                                                                    
VeloCty a été évalué avec des programmes de test de performance, 17 écrits en MATLAB et 9 écrits en Python.                              
Les résultats de VeloCty en utilisant toutes nos optimisations sur les                              
tests en MATLAB montrent qu'il est 1.3 à 458 fois plus rapide que                                   
l'interpréteur et le compilateur en-ligne de MATLAB 2014b par                                       
MathWorks.  Pour les tests en Python, VeloCty est 44.11 à 1681 fois                                 
plus rapide que l'interpréteur CPython. 
\end{otherlanguage}

