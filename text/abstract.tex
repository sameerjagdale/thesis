High-level scientific languages such as \matlab and Python's NumPy 
library are gaining popularity among scientists and 
mathematicians. These languages provide many features such as dynamic typing, 
high-level scientific functions etc. which allow easy 
prototyping. However these features also inhibit performance of the code. 

We present \velocty, an optimizing static compiler for \matlab and Python as a 
solution to the problem of enhancing performance of programs written in these 
languages. In most programs, a large portion of the time is spent executing a 
small part of the code. Moreover, these sections can often be compiled ahead of 
time and improved performance can be achieved by optimizing only these `hot' 
sections of the code. \velocty takes as input functions written in \matlab and 
Python specified by the user and generates an equivalent C++ version. \velocty also generates glue code to interface with \matlab and Python. The generated code can then be compiled and packaged as 
a shared library that can be linked to any program written in \matlab and 
Python. We also implemented optimisations to eliminate array bounds checks, 
reuse previously allocated memory during array operations and support parallel execution using OpenMP. 

\velocty uses the Velociraptor toolkit. We implemented a C++ backend for the 
Velociraptor intermediate representation, VRIR, and language-specific runtimes 
for \matlab and Python. We have also implemented a \matlab VRIR generator using the \mclab toolkit. 

\velocty was evaluated using 17 \matlab benchmarks and 9 Python benchmarks. The \matlab benchmark versions 
compiled using \velocty with all optimisations enabled were between 1.3 to 458 times faster than the MathWorks' 
\matlab2014b interpreter and JIT compiler. Similarly, Python benchmark versions were between 44.11 and 1681 times faster than the CPython interpreter.
