The primary goal of the thesis was to ensure correct compilation of code from \matlab and Python to C++. An additional goal was to improve the performance of the generated code. Initial experiments showed that turning on bounds check slowed down 6 of the 17 benchmarks and 3 of the 9 benchmarks in Python. The geometric mean of the slowdown compared to bounds check turned off was 3.66 for \matlab and 1.63 for Python. Additionally, while analysing the generated code, we found that array operations being performed inside loops were allocating memory to the same output array for every iteration. We determined that by optimising the code to eliminate bounds checks and unnecessary memory allocations, we gain a significant improvement in performance. In this chapter we first discuss the bounds check implementation followed by the two optimisations, namely elimination of bounds checks and elimination of redundant memory allocations. 

\section{Bounds Checks}
Scientific languages like \matlab and Python support array bounds checks for indexing operations. These checks ensure that the program does not crash abruptly and instead throws an error before exiting. On the other hand, C++ does not implicitly support array bounds checks. Hence we provide bounds checks through the runtime library.

Due to differences in semantics of \matlab and Python, the bounds check implementations for both languages are different. Hence we provide different implementations for the two languages. Array growth is also carried out by the bounds check implementation for languages that support it. 

However, the API for both language implementations is the same. The entry point function for bounds checks is a templated function called \textsf{checkBounds}. Listing \ref{lst:boundscheck} gives an example of the bounds check function for the array \textsf{c}. The bounds check functions are called inside conditional blocks which allows the user to turn the checks on or off while compiling the code. The first parameter is the reference to array on which the indexing operation is performed. The second parameter is a Boolean flag which is set to true if the Boolean operation is on the LHS of an assignment statement. This flag is used to determine if the array should be grown when one or more indices exceed bounds. This check is only used by the \matlab implementation. The third parameter is the number of indices. This parameter is required since the function accepts variable arguments. The remaining parameters are the indices which are passed as VrIndex structs. Passing indices as VrIndex structs allows the function to handle different index types such as ranges and arrays.

\begin{lstlisting}[language=c,caption={An example of the bounds check function call.},label={lst:boundscheck}]
//Bounds check 
#ifdef BOUND_CHECK
	checkBounds<VrArrayPtrF64,double>(&c,false,2,vrIndex(i),vrIndex(j));
#endif
\end{lstlisting}


However, the default bounds check function performs poorly due to dynamic memory allocation. The implementation inserts the indices into an array and performs checks while iterating over the array. Using an array simplifies the code for the checks. However, since the number of indices can vary, the array cannot be created at compile time.

In order to improve performance of the bounds checks, we implemented specialised versions of the bounds check function for index operations with one, two and three indices. We also implemented three additional versions for index operations where all indices have numeric values. These specialised functions are called checkBounds\_spec. Listing \ref{lst:boundscheck_spec} shows an example of the specialised version of the bounds check function. The function is specialised for two indices, both of which have numeric values. The first two parameters denote the array and whether the operation is performed on the LHS like in the default function. The remaining two parameters are the indices. 
\begin{lstlisting}[float,language=c,caption={An example of the specialised bounds check function call},label={lst:boundscheck_spec}]
//Bounds check 
#ifdef BOUND_CHECK
	checkBounds_spec<VrArrayPtrF64,double>(&c,false,
		static_cast<dim_type>(i),static_cast<dim_type>(j));
#endif
\end{lstlisting}
\section{Bounds Check Elimination}
Slowdown when the bounds checks was identified to be higher when the checks are performed inside loop bodies. In such cases, the checks are performed for every loop iteration resulting in the slowdown. Consider the example given in Listing \ref{lst:example_optim}. The example contains an index expression on the array y inside a loop. The index expression consists of a single index (k-1). The loop has a starting value of 1 and a final value of n-1. The value of the loop variable k is incremented by one with every iteration. As we can see, the index is a linear function of the loop variable k. The loop bounds are not modified inside the loop. Moreover, the step value of the loop is a constant and hence it can be inferred that the loop direction is upwards, that is, the loop iterates from a smaller start value to a larger stop value. By replacing the loop variable by the start and stop expressions of the loop, we get the lower and upper bounds of the index respectively. Thus the smallest value of the index inside the loop will be (1-1), that is, 0 and the largest value of the index will be ((n-1) - 1), that is, (n-2). Since we know what the lower and upper bound of the index, we can check whether these values exceed the array size or are less than the lowest index value supported by indexing scheme, outside the loop. If the index is valid, there is no need to perform bounds checks inside the loop. Thus this technique would improve the performance of the program. We use this optimisation technique, on a subset of the indices known as affine indices. 
\begin{lstlisting}[float,language=c,caption={Example C++ for loop with array index expressions},label={lst:example_optim}]
   for(k=1;k<=(n-1);k=k+1)
       {
  #ifdef BOUND_CHECK
          checkBounds_spec<VrArrayPtrF64,double>(&y,false,
				static_cast<dim_type>(k-1));
  #endif
          vr_temp12 = VR_GET_DATA_F64(y)[(k - 1)];
      }
\end{lstlisting}
\subsection{Affine indices}
A function of one or more variables is considered to be affine if it can be expressed as a sum of constant and constant multiples of the variables. Equation \ref{eq:affineIndex} gives a mathematical representation of affine functions. 
\begin{equation}
\label{eq:affineIndex}
f = C_0 + \sum\limits_{i=1}^n C_iX_i 
\end{equation}
where $C_i$ is the $i^{th}$ constant and $X_i$ is the $i^{th}$ variable. 

Affine indices can be defined as array indices which are affine functions of the loop induction variables. Table \ref{tab:affineIndex} gives examples of affine and non-affine array indices. 
\begin{table}[htbp]
\centering
\begin{tabular}{|c|c|}
\hline
Array Index & Affine \\ \hhline{|=|=|}
A(2*i+1)    & Yes    \\ \hline
A(i-1)    & Yes    \\ \hline
A(i*j)      & No    \\ \hline
A(i*i)      & No     \\ \hline
A(b(i))     & No     \\ \hline
\end{tabular}
\caption{Examples of affine and non-affine indices}
\label{tab:affineIndex}
\end{table}
\subsection{Technique}

The process of moving the checks outside the loop body can be divided into two parts. Identifying index operations which can be moved outside the loop and generating a if condition for the checks and two versions of the loop body.   

\subsubsection{Identifying valid index operations}
We define valid index operations to have the following properties. 
\begin{enumerate}
\item All indices should be affine functions of the loop variables. 
\item All the loop variables should have loop invariant bounds. 
\end{enumerate}
In order to determine whether a check for an index operation can be moved outside the loop body, we check whether individual indices are affine. Indices in VRIR are represented by IndexStructs. We do not consider indices which are ranges or expressions with non-scalar types. For indices with scalar expressions, we recursively traverse the expression until we reach a name or a constant expression or we reach an unsupported expression. Constant expressions are considered affine. In  case of name expressions we check whether the expression is a loop invariant or a loop variable. If the expression is a loop variable, we check whether the loop bounds are loop invariant. Apart from constant and name expressions, we also support binary and unary expressions. We only support a specialised case of the mult expression where the LHS and RHS  of the expression are either name or constant expressions. If both the LHS and the RHS are name expressions, they should both not be loop variables. We support the same expressions for checking the validity of loop bounds. The set of supported expressions are given in Table \ref{tab:affineIndexCheck}. 
\begin{table}[htbp]
\centering
\begin{tabular}{|c|c|}
\hline
Expression Name & Description        \\ \hhline{|=|=|}
plus            & Scalar Addition    \\ \hline
minus           & Scalar Subtraction \\ \hline
name            & Variable Name      \\ \hline
negate          & Unary Minus        \\ \hline
const           & Constant Value     \\ \hline
\end{tabular}
\caption{List of supported expressions for affine index check}
\label{tab:affineIndexCheck}
\end{table} 
\subsubsection{Generating code}
To implement the optimisation, the compiler generates an if statement. The if condition contains the checks for the valid index operations. For every index operation, we perform two checks. One for the lower bounds and another for the upper bounds. These checks are performed through functions called checkDimStart for the lower bounds and checkDimStop for the upper bounds. The functions takes as input integer indices. VrIndex structs are not required since indices containing ranges or having non-scalar types are not considered to be valid by the analysis. These functions are implemented inside the runtime library. The specialised functions for one two and three indices are also implemented in the library. Listing \ref{lst:boundsoptim} gives an example of the default and specialised functions. The default functions take the array name, the number of indices and the indices as parameters. The specialised functions named <default function name>\_spec take the array name and the indices as parameters. In the example, the functions take 2 indices as input. The loop variables are replaced by the lower and upper bounds of the loops when being passed as arguments to the check functions.We use the loop direction to determine whether the loop variables need to be replaced by the lower bounds for checkDimStart and upper bounds for checkDimStop or vice versa. If loop direction is up, that is the lower bound value is smaller than the upper bound value, the lower bound are used in checkDimStart and upper bound for checkDimStop. 

Listing \ref{lst:boundsoptimBlock} gives an example of the if statement generated by the compiler. The example shows a total of 6 functions, three for the lower bounds and three for the upper bounds of three arrays A, B and c. If the checks return true, a checks free version of the code is executed else the default version with checks is turned on is executed. 
\begin{lstlisting}[float,language=c,caption={An example of the default and specialised function calls for the boundscheck optimisations},label={lst:boundsoptim}]
//Default function
checkDimStart<VrArrayPtrF64>(c,2,1,1)
checkDimStop<VrArrayPtrF64>(c,2,m,n)

//Specialised function
checkDimStart_spec<VrArrayPtrF64>(c,1,1)
checkDimStop_spec<VrArrayPtrF64>(c,m,n)
\end{lstlisting}
\begin{lstlisting}[float,language=c,caption={An example of the if statement generated for the boundscheck optimisations},label={lst:boundsoptimBlock}]
 if(checkDimStart_spec<VrArrayPtrF64>(c,1,1) && checkDimStop_spec<VrArrayPtrF64>(c,m,n) && 
		checkDimStart_spec<VrArrayPtrF64>(B,1,1) && checkDimStop_spec<VrArrayPtrF64>(B,k,n) && 
		checkDimStart_spec<VrArrayPtrF64>(A,1,1) && checkDimStop_spec<VrArrayPtrF64>(A,m,k)) {
			<For Statements without bounds check >	
		} else {
			<For Statements with bounds check >	
		}
	
\end{lstlisting}
\section{Eliminating unnecessary memory allocations}
\label{sec:memoptimise}
Array operations and array slicing are implemented through functions in the runtime library. The output of these operations is written to a new array created inside the functions. Many times these operations are performed inside loops and the output is assigned to the same array variable. However, runtime memory allocation in expensive. Consider the example in Listing \ref{lst:memOptimiseExample}. The example contains an array operation, scal\_minus, which subtracts a scalar vr\_temp28  from every element in the array Rx and assigns it to an array that is created inside the function. The output of this operation is assigned to another array drx. Since this function is called inside the loop, a new array would be created on every loop iteration. However output array that is created will always be assigned to drx. The number of memory allocations could be reduced by reusing the memory that was assigned to drx during the first iteration for the subsequent iterations. Memory can only be reused if the size of drx is greater than or equal to the output of scal\_minus. Hence, the functions will have to be modified to perform a check for ensuring memory can be reused. Moreover, the function signature will have to to be modified to add a reference to the array to which the output is assigned, in this case, drx. Another alternative and is to implement a specialised function for the optimisation which satisfies the above mentioned criterion. We chose the second alternative for the optimisation. 
\begin{lstlisting}[float,language=c,caption={An example of an array operation which is optimised },label={lst:memOptimiseExample}]
for(k=1;k<= n;k=k+static_cast<long>(1)) {
	drx = BlasDouble::scal_minus(Rx,vr_temp28);
}
\end{lstlisting}

\subsection{Supported Functions}
Since a check for sufficient memory allocation needs to be made inside the function, a reference to the output array also needs to the be passed. Hence we implement specialised functions for this optimisation. The Supported library functions include many of the  array operations described in Subsection \ref{subsec:arrayOps} and a few other library functions. For dimension collapsing functions we support cases where a scalar value is returned. Table \ref{tab:memoptimiselist} gives a list of functions for which an implementation support the memory optimisation exists. 
\begin{table}[htbp]
\centering
\begin{tabular}{|c|c|c|}
\hline
Function Name & Function description         & Scalar version \\ \hhline{|=|=|=|}
mmult         & Matrix multiplication        & Yes            \\ \hline
scal\_mult    & Scalar Matrix Multiplication & No             \\ \hline
vec\_add      & Array Addition              & No             \\ \hline
vec\_copy     & Array Copy                   & No             \\ \hline
vec\_sub      & Array Subtraction            & No             \\ \hline
scal\_add     & Scalar Array Addition        & No             \\ \hline
scal\_minus   & Scalar Array Subtraction     & No             \\ \hline
transpose     & Matrix Transpose             & No             \\ \hline
sum           & Sum of Array Elements        & Yes            \\ \hline
mean          & Mean of Array Elements       & Yes            \\ \hline
sliceArray          & Get array slice        & No            \\ \hline
\end{tabular}
\caption{List of functions that support memory optimisation}
\label{tab:memoptimiselist}
\end{table}
\subsection{Checking for Sufficient Memory}
As mentioned before, the specialised functions accept a reference to the output array as an input parameter. The output array is then checked to determine whether the maximum number of elements that the array can hold is greater than or equal to the number of elements of the output of the operation performed by the function. The number of elements are calculated by taking the product of the dimensions of the array. If the memory is sufficient, no memory is allocated to the array whereas memory is allocated if it is not sufficient. In either case, the dimensions are modified to be equal to the expected dimensions of the output of the array operation.   
\subsection{Code Generation}

\begin{table}
  \begin{tabular}{|l|l|}
  \hline
  Array operation without optimisation & 
Array operation with optimisation 
   \\
  \hhline{|=|=|}
  
  {
  \begin{lstlisting}[language=c,frame=none, numbers=none]
	drz = 
		BlasDouble::scal_minus(Rz, vr_temp30);
  \end{lstlisting}
  } &
   {
   \begin{lstlisting}[language=c,frame=none, numbers=none]
	 BlasDouble::scal_minus(Rz, 
		vr_temp30,&drz);
   \end{lstlisting}
   } \\
   \hline
   \end{tabular}
   \caption[Generated code with and without memory optimisations]{Table shows the generated code with and without memory optimisations}
   \label{tab:memOptimise}
   \end{table}
While generating code for assignment statements, the compiler checks for library call expressions which can be compiled to specialised function calls. The compiler does this by checking the function name against a hash set which stores a list of functions that can be specialised. The compiler generates the specialised function call in place of the assignment statement. It then passes the reference LHS of the assignment statement as a parameter to the function. 

Table \ref{tab:memOptimise} gives an example of the generated codes with and without the memory optimisations. The left column shows the function call without the optimisation and the right column shows the function call with the optimisation. The example shows an array operation scal\_minus which subtracts a scalar vr\_temp30 from every element in the array Rz and assigns the result to drz. The optimisation passes the output array drz's reference to the function where a check for sufficient memory is performed.
