Over the years, many dynamic languages were developed which were either developed for scientific computing or provided libraries to implement the same. This list includes but is not restricted to \matlab and NumPy. Additionally, improving the performance of dynamic languages through ahead of time compilation or just in time compilation has been the interest of researchers for many years. Hence many projects, industry-based, academic and community-based , have been implemented. In this chapter, we dicuss some languages which can be viewed as alternatives to \matlab and NumPy, followed by different tools which aim to improve the performance \matlab and NumPy.
\section{Alternatives to \matlab and NumPy}
A few examples of open-source alternatives to \matlab and NumPy are, Julia\cite{julia}, Scilab\cite{scilab}, R\cite{rlang} and Octave\cite{octave}. Julia is a high-performance dyanamic langauge for high-performance computing. Julia supports distributed-parallel execution and a high-performance library for numerical computing. Scilab is an open source software for numerical computing. Scilab supports high-level 2D and 3D visualisation functions among others. R is a language for statistical computing. R is an alteranate implementation of the S language. Octave is an open-source implementation of \matlab. It supports most of the \matlab code. Although it was previously interpreted, a JIT compiler was added in version 3.8.1.
\section{Tools for NumPy}
\subsection{Cython}
Cython is a programming language for writing C extensions for Python. Cython was originally based on the Pyrex project\cite{pyrex}.  Cython allows optional static declarations. This allows C semantics, which are static and fast to be applied for parts of the code instead of dynamic Python semantics. Cython generated code can also make calls into C libraries. Listing \ref{lst:cython} gives an example of the Cython code for the benchmark arc\_distance. Types are provided using the \textsf{cdef} prefix. 

\begin{lstlisting}[language=python, label={lst:cython}, caption={ The Cython code with static type annotations that will is taken as input by Cython to generate C code. The example is of the arc\_distance benchmark}]

def pairwise_kernel(np.ndarray[double,ndim=2] data):
cdef int n_samples = data.shape[0]
cdef int n_features = data.shape[1]
cdef double tmp, d
cdef np.ndarray[double,ndim=2] distances = np.empty((n_samples, n_samples))
for i in range(n_samples):
  for j in range(n_samples):
	  d = 0.0
	  for k in range(n_features):
		  tmp = data[i, k] - data[j, k]
		  d += tmp * tmp
	  distances[i, j] = sqrt(d)
return distances
\end{lstlisting}

Unlike \velocty, Cython can only generate C code from Python. Moreover, Cython inserts additional checks for types and memory management. \velocty on the other hand assumes that all type annotations provided are correct and hence does not place checks.
\subsection{Numba}
Numba\cite{numba} is a library for Python that can perform JIT compilation given a few annotations. Numba generates machine code using the LLVM\cite{Lattner:2004:LCF:977395.977673} infrastructure. Numba also has a CUDA\cite{Nickolls:2008:SPP:1365490.1365500} backend which is currently experimental. Numba can also perform static compilation using the pycc tool that is provided with the library. Numba is a comparitively new project with the initial release in 2012. 
\subsection{Theano}
Theano\cite{Bastien-Theano-2012} is a Python library that allows users to define mathematical expressions and then optimises and generates C code dynamically. Theano combines various aspects of computer algebra systems with an optimising compiler. It also generates CUDA code for GPUs. Theano has tight integeration with NumPy.
\section{\matlab Tools}
\subsection{\matlab-coder}
The Mathworks' \matlab-coder\cite{matlab-coder} is a tool to compile \matlab functions to C/C++. \matlab-coder accepts the \matlab function as well as types and shapes for the input arguments of the array and generates C/C++ code. \matlab-coder offers 3 different options for compilation. The user can generate a standalone C/C++ executable, a C/C++ shared library or a MEX\cite{mex} function that can be called from \matlab. Similar to Cython, \matlab-coder adds type and memory checks which are added by \velocty.
\subsection{Falcon}
The Falcon project\cite{DeRose:1999} is a \matlab to Fortran90 compiler. Falcon implements type inference algorithms that were developed for the APL\cite{Budd:1983:ACU:390005.801218,Joisha:2000:CDI:570406.570408,Weiss:1981:CTS:390007.805380} language and the SETL\cite{Schwartz:1975:ADS:512976.512981} language. The compiler inlines all functions and scripts into a single function. Falcon uses static single assignment(SSA) based intermediate representation. Additionally they also collected a set of benchmarks for their experiments. We use many of these benchmarks. 
\subsection{MAJIC}
MAJIC stands for \matlab Just in Time Compiler. It is a continuation of the Falcon project. It performs three different types of optimisations: Source level optimisations for matrix operations, JIT compilation, so that the effects of the 'wild` \matlab features are reduced and lastly specialised optimsations for sparse matrices. 
\subsection{MENHIR}
MENHIR\cite{Chauveau:1999} is a retargetable compiler for \matlab that can compile either C or Fortran given a target system description(MTSD). An efficient code is generated that exploits optimised sequential and parallel libraries. The MTSD is used to generate the optimised code for a specific platform. 
\subsection{Mc2For}
Mc2For\cite{Li:2014} is a source to source compiler which transforms \matlab code to equivalent Fortran code. Mc2For was also developed using the \mclab framework. Even though Mc2For is a ahead of time compiler similar to \velocty, Mc2For compiles complete \matlab programs instead of 'hot` code sections. Hence if any part of the \matlab program has constructs that cannot be compiled ahead of time, the entire program cannot be compiled using Mc2For. 
\subsection{MiX10}
MiX10\cite{vkumar14} is a source to source compiler which compiles \matlab code to X10\cite{x10}. X10 is a high-performance language developed at IBM. MiX10 compiles dynamically typed \matlab code to a statically typed X10 language. The X10 compiler itself compiles the X10 code to C++ and Java. Similar to Mc2For compiles complete \matlab programs to X10 and hence cannot compile \matlab programs having dyanamic constructs.

