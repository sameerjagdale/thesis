An important contribution of the thesis is the static generation of C++ code from VRIR. Due to differences in the semantics of VRIR and C++, we faced various challenges during code generation. As described in \chapref{chap:Background}, VRIR is a high level strongly typed AST designed to support easy compilation of a wide range of Array based languages. Hence, it supports different indexing schemes such as 0-indexing, 1-indexing and negative indexing as well as different array layout schemes such as row-major and column-major. C++ on the other hand does not have an built in support for arrays, and only supports 0-indexing and a row major layout. Moreover, VRIR also supports multiple returns. On the other hand, we can only return a single value, which can be a scalar, class, struct or a pointer, in C++. This chapter describes how different nodes in VRIR are mapped to C++ constructs including those which C++ does not implicitly support.
\section{Mapping Types}
Data types in VRIR, known as VTypes, can be categorised into 5 types :
\subsection{Scalar Type}
The scalar type is used define the primitive data type.
Different types of Scalar values are Int32, Int64, Float32, Float64 and Bool.
The mapping of VTypes to different C++ types is shown in table \ref{tab:typeMap}.
\begin{table}[h]
\centering
\begin{tabular}{|c|c|c|c|}
\hline
\multicolumn{2}{|c|}{Scalar Type}                                                                 & \multirow{2}{*}{Real / Complex}  & \multirow{2}{*}{C++ types} \\ \cline{1-2}
Name                                         & S-Expression                                      &                              &                            \\ \hline
\multirow{2}{*}{Float32}                     & \multicolumn{1}{c|}{\multirow{2}{*}{( float32 )}} & REAL                         & float                        \\ \cline{3-4} 
                                             & \multicolumn{1}{c|}{}                             & \multicolumn{1}{c|}{COMPLEX} & float complex              \\ \hline
\multirow{2}{*}{Float64}                     & \multicolumn{1}{c|}{\multirow{2}{*}{( float64 )}} & REAL                         & double                       \\ \cline{3-4} 
                                             & \multicolumn{1}{c|}{}                             & \multicolumn{1}{c|}{COMPLEX} & double complex             \\ \hline
\multicolumn{1}{|c|}{\multirow{2}{*}{Int32}} & \multirow{2}{*}{( int32 )}                        & REAL                         & int                      \\ \cline{3-4} 
\multicolumn{1}{|c|}{}                       &                                                   & \multicolumn{1}{c|}{COMPLEX} & Not Supported                          \\ \hline
\multicolumn{1}{|c|}{\multirow{2}{*}{Int64}} & \multirow{2}{*}{( int64 )}                        & REAL                         & long                     \\ \cline{3-4} 
\multicolumn{1}{|c|}{}                       &                                                   & \multicolumn{1}{c|}{COMPLEX} & Not Supported                          \\ \hline
\multicolumn{1}{|c|}{\multirow{2}{*}{Bool}} & \multirow{2}{*}{( bool )}                        & REAL                         & bool                     \\ \cline{3-4} 
\multicolumn{1}{|c|}{}                       &                                                   & \multicolumn{1}{c|}{COMPLEX} & Not Supported                          \\ \hline
\end{tabular}
\caption[VType to C++ type mapping]{VType to C++ type mapping. The tables shows the different C++ will be mapped to from the VTypes.} 
\label{tab:typeMap}
\end{table}
\subsection{Array Types}
\begin{table}[htbp]
\centering
\begin{tabular}{|c|c|c|c|}
\hline
\multicolumn{3}{|c|}{Array Type}                                                                                                                                                                                  & \multirow{2}{*}{VrArray}         \\ \cline{1-3}
Name                                                                       & Real / Complex & S-expression                                                                                                            &                                  \\ \hline
\multirow{2}{*}{\begin{tabular}[c]{@{}c@{}}Float64\\ Array\end{tabular}}   & REAL       & \begin{tabular}[c]{@{}c@{}}( arraytype :ndims :layout \\ ( float64 :ctype complex) )\end{tabular}                       & VrArrayF64                       \\ \cline{2-4} 
                                                                           & COMPLEX    & \multicolumn{1}{l|}{\begin{tabular}[c]{@{}l@{}}( arraytype :ndims :layout \\ ( float64 :ctype complex) )\end{tabular}}  & \multicolumn{1}{c|}{VrArrayCF64} \\ \hline
\multirow{2}{*}{\begin{tabular}[c]{@{}c@{}}Float32 \\ Array\end{tabular}}  & REAL       & \begin{tabular}[c]{@{}c@{}}( arraytype :ndims \\ :layout \\ ( float32  ctype: real ) )\end{tabular}                     & VrArrayF32                       \\ \cline{2-4} 
                                                                           & COMPLEX    & \multicolumn{1}{l|}{\begin{tabular}[c]{@{}l@{}}( arraytype :ndims :layout \\ ( float32,ctype: complex ) )\end{tabular}} & \multicolumn{1}{l|}{VrArrayCF32} \\ \hline
\begin{tabular}[c]{@{}c@{}}Int32 \\ Array\end{tabular}                     & REAL       & \begin{tabular}[c]{@{}c@{}}( arraytype :ndims :layout \\ ( int32 ctype: real ) )\end{tabular}                           & VrArrayI32                       \\ \hline
\begin{tabular}[c]{@{}c@{}}Int64 \\ Array\end{tabular}                     & REAL       & \begin{tabular}[c]{@{}c@{}}( arraytype :ndims :layout \\ ( int32 ctype: real ) )\end{tabular}                           & VrArrayI64                       \\ \hline
\multicolumn{1}{|c|}{\begin{tabular}[c]{@{}l@{}}Bool\\ Array\end{tabular}} & REAL       & \multicolumn{1}{l|}{\begin{tabular}[c]{@{}l@{}}( arraytype :ndims :layout \\ ( int32 ctype: real ) )\end{tabular}}      & \multicolumn{1}{c|}{VrArrayB}    \\ \hline
\end{tabular}
\caption[Array Types]{ArrayType map. The table shows the VrArray types the ArrayTypes in VRIR are mapped to.}
\label{tab:arrayTypeMap}
\end{table}

\begin{table}[htbp]
\centering
\begin{tabular}{|c|c|}
\hline
VrArray Type & Data Field Type \\ \hline
VrArrayF64   & double          \\ \hline
VrArrayF32   & float           \\ \hline
VrArrayCF64  & double complex  \\ \hline
VrArrayCF32  & float complex  \\ \hline
VrArrayI64   & long            \\ \hline
VrArrayI32   & int             \\ \hline
VrArrayB     & bool            \\ \hline
\end{tabular}
\caption[Data field types of different VrArrays]{Data field types of different VrArrays. Table depicting the types of the data field for different VrArray types.}
\label{tab:arrayDataMap}
\end{table}
Unlike array-based languages, C++ arrays do not store additional information such as the number of dimensions or the sizes of each dimensions. This information is useful while performing various operations such as multiplication, addition etc and hence it was necessary for us to store it. One solution was to take store this information separately. However, this approach increases the number of parameters that need to be passed to functions implementing array operations. Moreover, when assigning to an array additional code that needs to be generated to update the dimension sizes and the number of dimensions. Hence, we implemented structs for arrays of all the data types supported by VRIR. We call the structs collectively as VrArrays. The equivalent VrArray types for different array types are shown in table \ref{tab:arrayTypeMap}.
\subsubsection{VrArrays}
VrArrays are represented as C++ structs and encapsulate array data as well as the meta-data. They contain a pointer to the data as well as other necessary information such as the number of dimensions and the size of each dimension. There are different VrArrays for different data types. The different VrArray types and the VRIR types from which they are mapped are given in table \ref{tab:arrayTypeMap}. The structure of the different VrArrays in given below. Each VrArray has a data field which is a pointer to the array data. The type of the data field depends on the type of the VrArray. For example, the type of VrArrayF64, which is used to represent an float64 array is double. 
There are separate VrArray types for complex and real arrays of the same type. All operations on arrays in the language runtime take VrArrays as input. This allows single parameter to be passed for array instead of passing the data, dimensions and number of dimensions separately. The full list of VrArrays and the types of their corresponding data fields is given in table \ref{tab:arrayDataMap}.
\begin{lstlisting}[language=c, label={vrArrayF64Struct}, caption={Structures of VrArrays for real data}]
typedef struct VrArrayF64{
  double *data;
  dim_type* dims;  
  int ndims;
}VrArrayF64;

typedef struct VrArrayF32{
  float *data;
  dim_type* dims;  
  int ndims;
}VrArrayF32;

typedef struct VrArrayI32{
  int *data;
  dim_type* dims;  
  int ndims;
}VrArrayI32;

typedef struct VrArrayI64{
  int *data;
  dim_type* dims;  
  int ndims;
}VrArrayI64;

\end{lstlisting}

\begin{lstlisting}[language=c, label={vrArrayF64Struct}, caption={Structures of VrArrays for complex data}]

typedef struct VrArrayCF32{
  float complex *data;
  dim_type* dims;  
  int ndims;
}VrArrayCF32;

typedef struct VrArrayCF64{
  double complex *data;
  dim_type* dims;  
  int ndims;
}VrArrayCF64;

\end{lstlisting}

\subsection{Void Type}
\begin{table}[htbp]
\centering
\begin{tabular}{|c|c|}
\hline
Void Type & Generated Void \\ \hline
( void)   & void           \\ \hline
\end{tabular}
\caption[voidTypeMapping]{The table describes the generated C++ code for a void type.}
\label{tab:voidTypeMap}
\end{table}
The void type is used in most cases inside a Func Type to convey the absence of either input or output parameters. The void type is mapped to a simple `void' in C++. Table \ref{tab:voidTypeMap} shows the mapping
\subsection{Tuple Type}
Tuple types are used to define data structures which can have data of different types. While generating C++ code, the tuple types are used to generate structs that are in turn used to support data structures containing heteregenous data.
\subsection{Func Type}
Func types are associated with function definitions and function handles. They contain information about the types of the input and out parameters of the function. The function types are used for generating function definition. More information about how function definitions are generated in given in section \ref{sec:functions}.
\section{Modules}
Modules are top-level constructs in VRIR. They contain one or more functions which have to be compiled to C++. A module also has an attribute called indexing which defines the indexing scheme. The indexing scheme can either be 0 or 1 indexing. More details about how the indexing attribute is used can be found in section \ref{sec:indexing}.
\section{Functions} 
\label{sec:functions}
\begin{table}[htbp]
\begin{tabular}{|l|l|}
\hline
\multicolumn{1}{|c|}{Func Type}                                                                                                                                                                                                                                       & \multicolumn{1}{c|}{Generated Function}                                                          \\ \hline
\begin{tabular}[c]{@{}l@{}} (function babai \\
   (functype \\
   (intypes \\
   ( arraytype :layout colmajor :ndims 2(float64 :ctype 0)) \\
	( arraytype :layout colmajor :ndims 2(float64 :ctype 0))) \\
   (outtypes \\
   ( arraytype :layout colmajor :ndims 2(float64 :ctype 0)))) \\
	(arglist \\
   (arg :id0 \\
   )(arg :id1 \\
   )) \\
	(body ... )
\end{tabular} & \begin{tabular}[c]{@{}l@{}}VrArrayPtrF64 \\ babai (VrArrayPtrF64 R,VrArrayPtrF64 y)\end{tabular} \\ \hline
\end{tabular}
\caption[funcTypeMap]{The table shows an example func type and the equivalent function signature that was generated using the func type.}
\label{tab:funcTypeMap}
\end{table}
Functions in VRIR are compiled to separate functions in C++. The function node in VRIR has multiple children all of which are required to generate the C++ code for the function. The list of children node of the function node and their role in the code generation process is as follows :
\begin{itemize}
\item Name : The function name represents the name of the function. A C++ function of the same name is generated.   
\item Arglist : The arglist is a list of integers which are the Ids of the input arguments in the symbol table.
\item Func type : The Func type is used for generating the input argument types and the return type.
\item Body : The body represents the body of the function. It consists of a list of statements. 
\end{itemize}
The table \ref{tab:funcTypeMap} depicts how a function node in VRIR is converted to a C++ function.
\subsection{Return types in VRIR}
\begin{lstlisting}[language=c,caption={Generated structure to handle multiple returns.},label={lst:multReturn}]
//Structure definition
typedef struct struct_adapt_ret { 
   VrArrayPtrF64 ret_data0;
   double ret_data1;
   double ret_data2;
   
   struct_adapt_ret(VrArrayPtrF64 ret_data0,double ret_data1,double ret_data2) :ret_data0(ret_data0),ret_data1(ret_data1),ret_data2(ret_data2)
   {
   }
   
}struct_adapt_ret;

//Function declaration
struct_adapt_ret adapt(double a,double b,double sz_guess,double tol);
\end{lstlisting}
C++ only permits single return types. On the other hand, VRIR supports multiple return types. In order to bridge this difference in semantics, we generate a struct definition whose fields are of the same types as the return types. A parameterized constructor is also provided to assign the different variables that need to be returned to the member fields of the struct. The structure definition and the function definition that returns the structure is shown in listing \ref{lst:multReturn}.
\section{Operators}
\label{sec:operators}
Arithemetic operators in VRIR, such as plus, minus, mult, div can only have scalar operands. Hence generating C++ code these operators in straightforward. They are mapped to the operators internally supported by C++. Thus plus is mapped to the `+' operator in C++, minus is mapped to `-'. The complete list is given in table \ref{tab:opMap}. \\
\begin{table}[h]
\centering
\begin{tabular}{|c|c|}
\hline
VRIR operators & C++ Operators \\ \hline
plus           & +             \\ \hline
minus          & -             \\ \hline
mult           & *             \\ \hline
div            & /             \\ \hline
and            & \&\&           \\ \hline
or             & ||            \\ \hline
lt             & \textless     \\ \hline
leq            & \textless=    \\ \hline
gt             & \textgreater  \\ \hline
geq            & \textgreater= \\ \hline
eq             & ==            \\ \hline
neq            & !=            \\ \hline
\end{tabular}
\caption[opMap]{VRIR operators to C++ operators Mapping. The table shows the C++ operators to which the VRIR operators are mapped.}
\label{tab:opMap}
\end{table}
Operations on arrays, on the other hand, are mapped to the LibCall expression in VRIR. Since C++ does not have internal operators for arrays we implemented function to support these operations on arrays. These functions are housed inside the language-specific runtime library. Where ever possible these functions made calls to BLAS functions for enchanced performance. In case of the \matlab\cite{matlab} runtime, we use the Intel Math Kernel Library\cite{mkl} or MKL implementation of BLAS and in case of Python\cite{python}, we use the openBLAS\cite{openblas} implementation.
\begin{table}[h]
\centering
\begin{tabular}{|c|c|c|c|}
\hline
VRIR Lib Call                               & Operand 1 & Operand 2 & C++ function \\ \hline
\multirow{2}{*}{Matrix Multiplication}      & Array     & Array     & mmult        \\ \cline{2-4} 
                                            & Array     & Scalar    & scal\_mult   \\ \hline
\multirow{2}{*}{Elementwise Multiplication} & Array     & Array     & vec\_mult    \\ \cline{2-4} 
                                            & Array     & Scalar    & scal\_mult   \\ \hline
\multirow{2}{*}{Matrix Left Division}       & Array     & Array     & mat\_ldiv    \\ \cline{2-4} 
                                            & Array     & Scalar    & scal\_div    \\ \hline
\multirow{2}{*}{Matrix Right Division}      & Array     & Array     & mat\_rdiv    \\ \cline{2-4} 
                                            & Array     & Scalar    & scal\_div    \\ \hline
\multirow{2}{*}{Elementwise Division}       & Array     & Array     & elem\_div    \\ \cline{2-4} 
                                            & Array     & Scalar    & scal\_div    \\ \hline
\multirow{2}{*}{Array Addition}             & Array     & Array     & vec\_add     \\ \cline{2-4} 
                                            & Array     & Scalar    & scal\_add    \\ \hline
\multirow{2}{*}{Array Subtraction}          & Array     & Array     & vec\_sub     \\ \cline{2-4} 
                                            & Array     & Scalar    & scal\_minus  \\ \hline
Array Copy                                  & Array     & Array     & vec\_copy    \\ \hline
Matrix Transpose                            & Array     & -         & transpose    \\ \hline
\end{tabular}
\caption[List of operations on Arrays]{The table shows the different C++ functions array operators are mapped to. }
\label{tab:arrayOpMap}
\end{table}
\section{Statements}
VRIR supports various statements such as assignments, for-loops, while-loops etc. Most of the statements are directly supported in C++. Table contains a list of VRIR statement types and their equivalent C++ types that are generated. The assignment, for and return statements have special cases which need to be supported.
\subsection{Assignment Statement}

\subsection{For Statement}
\subsection{Return Statement}
\section{Expressions}
\section{Basic Indexing}
\label{sec:indexing}
\subsection{ Generating a single index value from multiple indices}
\subsection{Negative Indexing}
\section{Advanced Indexing}
\subsection{VrIndex}
\subsection{Array Slicing}
\section{Symbol Table}
\section{Run time library}
