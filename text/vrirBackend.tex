An important contribution of the thesis is the static generation of C++ code from VRIR. Due to differences in the semantics of VRIR and C++, we faced various challenges during code generation. As described in \chapref{chap:Background}, VRIR is a high level strongly typed AST designed to support easy compilation of a wide range of Array based languages. Hence, it supports different indexing schemes such as 0-indexing, 1-indexing and negative indexing as well as different array layout schemes such as row-major and column-major. C++ on the other hand does not have an in-built support for arrays, and only supports 0-indexing and a row major layout. Moreover, VRIR also supports multiple returns. On the other hand, we can only return a single value, which can be a scalar, class, struct or a pointer, in C++. This chapter describes how different nodes in VRIR are mapped to C++ constructs including those which C++ does not implicitly support.
\section{Mapping types}
\subsection{VRIR types}
Data types in VRIR, known as VTypes, can be categorised into 5 types :
\subsection{Scalar type}
The scalar type is used define the primitive data type.
Different types of Scalar values are Int32, Int64, Float32, Float64 and Bool.
The mapping of VTypes to different C++ types is shown in table \ref{tab:typeMap}.
\begin{table}[h]
\centering
\begin{tabular}{|C{3cm}|C{3cm}|}
\hline
VTypes  & C++ types \\ \hline
Int32   & int       \\ \hline
Int64   & long      \\ \hline
Float32 & float     \\ \hline
Float64 & double    \\ \hline
Bool    & bool      \\ \hline
\end{tabular}
\caption[VType to C++ type mapping]{VType to C++ type mapping. The tables shows the different C++ types in the column on the right will be mapped to from the VTypes in the right.} 
\label{tab:typeMap}
\end{table}
\subsection{Array Types}
\section{Operators}
\section{VrArrays}
\section{Statements}
\subsection{Assignment statement}
\subsection{For Statement}
\section{Basic Indexing}
\subsection{ Generating single index value from multiple indices}
\subsection{negative indexing}
\section{Advanced indexing}
\subsection{VrIndex}
\subsection{Array Slicing}
