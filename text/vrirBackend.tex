An important contribution of the thesis is the static generation of C++ code from VRIR. Due to differences in the semantics of VRIR and C++, we faced various challenges during code generation. As described in \chapref{chap:Background}, VRIR is a high level strongly typed AST designed to support easy compilation of a wide range of Array based languages. Hence, it supports different indexing schemes such as 0-indexing, 1-indexing and negative indexing as well as different array layout schemes such as row-major and column-major. C++ on the other hand does not have an built in support for arrays, and only supports 0-indexing and a row major layout. Moreover, VRIR also supports multiple returns. On the other hand, we can only return a single value, which can be a scalar, class, struct or a pointer, in C++. This chapter describes how different nodes in VRIR are mapped to C++ constructs including those which C++ does not implicitly support.
\section{Mapping Types}
Data types in VRIR, known as VTypes, can be categorised into 5 types :
\subsection{Scalar Type}
The scalar type is used define the primitive data type.
Different types of Scalar values are Int32, Int64, Float32, Float64 and Bool.
The mapping of VTypes to different C++ types is shown in table \ref{tab:typeMap}.
\begin{table}[h]
\centering
\begin{tabular}{|c|c|c|c|}
\hline
\multicolumn{2}{|c|}{Scalar Type}                                                                 & \multirow{2}{*}{Real / Complex}  & \multirow{2}{*}{C++ types} \\ \cline{1-2}
Name                                         & S-Expression                                      &                              &                            \\ \hline
\multirow{2}{*}{Float32}                     & \multicolumn{1}{c|}{\multirow{2}{*}{( float32 )}} & REAL                         & float                        \\ \cline{3-4} 
                                             & \multicolumn{1}{c|}{}                             & \multicolumn{1}{c|}{COMPLEX} & float complex              \\ \hline
\multirow{2}{*}{Float64}                     & \multicolumn{1}{c|}{\multirow{2}{*}{( float64 )}} & REAL                         & double                       \\ \cline{3-4} 
                                             & \multicolumn{1}{c|}{}                             & \multicolumn{1}{c|}{COMPLEX} & double complex             \\ \hline
\multicolumn{1}{|c|}{\multirow{2}{*}{Int32}} & \multirow{2}{*}{( int32 )}                        & REAL                         & int                      \\ \cline{3-4} 
\multicolumn{1}{|c|}{}                       &                                                   & \multicolumn{1}{c|}{COMPLEX} & Not Supported                          \\ \hline
\multicolumn{1}{|c|}{\multirow{2}{*}{Int64}} & \multirow{2}{*}{( int64 )}                        & REAL                         & long                     \\ \cline{3-4} 
\multicolumn{1}{|c|}{}                       &                                                   & \multicolumn{1}{c|}{COMPLEX} & Not Supported                          \\ \hline
\multicolumn{1}{|c|}{\multirow{2}{*}{Bool}} & \multirow{2}{*}{( bool )}                        & REAL                         & bool                     \\ \cline{3-4} 
\multicolumn{1}{|c|}{}                       &                                                   & \multicolumn{1}{c|}{COMPLEX} & Not Supported                          \\ \hline
\end{tabular}
\caption[VType to C++ type mapping]{VType to C++ type mapping. The tables shows the different C++ will be mapped to from the VTypes.} 
\label{tab:typeMap}
\end{table}
\subsection{Array Types}
\begin{table}[htbp]
\centering
\begin{tabular}{|c|c|c|c|}
\hline
\multicolumn{3}{|c|}{Array Type}                                                                                                                                                                                  & \multirow{2}{*}{VrArray}         \\ \cline{1-3}
Name                                                                       & Real / Complex & S-expression                                                                                                            &                                  \\ \hline
\multirow{2}{*}{\begin{tabular}[c]{@{}c@{}}Float64\\ Array\end{tabular}}   & REAL       & \begin{tabular}[c]{@{}c@{}}( arraytype :ndims :layout \\ ( float64 :ctype complex) )\end{tabular}                       & VrArrayF64                       \\ \cline{2-4} 
                                                                           & COMPLEX    & \multicolumn{1}{l|}{\begin{tabular}[c]{@{}l@{}}( arraytype :ndims :layout \\ ( float64 :ctype complex) )\end{tabular}}  & \multicolumn{1}{c|}{VrArrayCF64} \\ \hline
\multirow{2}{*}{\begin{tabular}[c]{@{}c@{}}Float32 \\ Array\end{tabular}}  & REAL       & \begin{tabular}[c]{@{}c@{}}( arraytype :ndims \\ :layout \\ ( float32  ctype: real ) )\end{tabular}                     & VrArrayF32                       \\ \cline{2-4} 
                                                                           & COMPLEX    & \multicolumn{1}{l|}{\begin{tabular}[c]{@{}l@{}}( arraytype :ndims :layout \\ ( float32,ctype: complex ) )\end{tabular}} & \multicolumn{1}{l|}{VrArrayCF32} \\ \hline
\begin{tabular}[c]{@{}c@{}}Int32 \\ Array\end{tabular}                     & REAL       & \begin{tabular}[c]{@{}c@{}}( arraytype :ndims :layout \\ ( int32 ctype: real ) )\end{tabular}                           & VrArrayI32                       \\ \hline
\begin{tabular}[c]{@{}c@{}}Int64 \\ Array\end{tabular}                     & REAL       & \begin{tabular}[c]{@{}c@{}}( arraytype :ndims :layout \\ ( int32 ctype: real ) )\end{tabular}                           & VrArrayI64                       \\ \hline
\multicolumn{1}{|c|}{\begin{tabular}[c]{@{}l@{}}Bool\\ Array\end{tabular}} & REAL       & \multicolumn{1}{l|}{\begin{tabular}[c]{@{}l@{}}( arraytype :ndims :layout \\ ( int32 ctype: real ) )\end{tabular}}      & \multicolumn{1}{c|}{VrArrayB}    \\ \hline
\end{tabular}
\caption[Array Types]{ArrayType map. The table shows the VrArray types the ArrayTypes in VRIR are mapped to.}
\label{tab:arrayTypeMap}
\end{table}

\begin{table}[htbp]
\centering
\begin{tabular}{|c|c|}
\hline
VrArray Type & Data Field Type \\ \hline
VrArrayF64   & double          \\ \hline
VrArrayF32   & float           \\ \hline
VrArrayCF64  & double complex  \\ \hline
VrArrayCF32  & float complex  \\ \hline
VrArrayI64   & long            \\ \hline
VrArrayI32   & int             \\ \hline
VrArrayB     & bool            \\ \hline
\end{tabular}
\caption[Data field types of different VrArrays]{Data field types of different VrArrays. Table depicting the types of the data field for different VrArray types.}
\label{tab:arrayDataMap}
\end{table}
Unlike array-based languages, C++ arrays do not store additional information such as the number of dimensions or the sizes of each dimensions. This information is useful while performing various operations such as multiplication, addition etc and hence it was necessary for us to store it. One solution was to take store this information separately. However, this approach increases the number of parameters that need to be passed to functions implementing array operations. Moreover, when assigning to an array additional code that needs to be generated to update the dimension sizes and the number of dimensions. Hence, we implemented structs for arrays of all the data types supported by VRIR. We call the structs collectively as VrArrays. The equivalent VrArray types for different array types are shown in table \ref{tab:arrayTypeMap}.
\subsubsection{VrArrays}
VrArrays are represented as C++ structs and encapsulate array data as well as the meta-data. They contain a pointer to the data as well as other necessary information such as the number of dimensions and the size of each dimension. There are different VrArrays for different data types. The different VrArray types and the VRIR types from which they are mapped are given in table \ref{tab:arrayTypeMap}. The structure of the different VrArrays in given below. Each VrArray has a data field which is a pointer to the array data. The type of the data field depends on the type of the VrArray. For example, the type of VrArrayF64, which is used to represent an float64 array is double. 
There are separate VrArray types for complex and real arrays of the same type. All operations on arrays in the language runtime take VrArrays as input. This allows single parameter to be passed for array instead of passing the data, dimensions and number of dimensions separately. The full list of VrArrays and the types of their corresponding data fields is given in table \ref{tab:arrayDataMap}.
\begin{lstlisting}[language=c, label={vrArrayF64Struct}, caption={Structures of VrArrays for real data}]
typedef struct VrArrayF64{
  double *data;
  dim_type* dims;  
  int ndims;
}VrArrayF64;

typedef struct VrArrayF32{
  float *data;
  dim_type* dims;  
  int ndims;
}VrArrayF32;

typedef struct VrArrayI32{
  int *data;
  dim_type* dims;  
  int ndims;
}VrArrayI32;

typedef struct VrArrayI64{
  int *data;
  dim_type* dims;  
  int ndims;
}VrArrayI64;

\end{lstlisting}

\begin{lstlisting}[language=c, label={vrArrayF64Struct}, caption={Structures of VrArrays for complex data}]

typedef struct VrArrayCF32{
  float complex *data;
  dim_type* dims;  
  int ndims;
}VrArrayCF32;

typedef struct VrArrayCF64{
  double complex *data;
  dim_type* dims;  
  int ndims;
}VrArrayCF64;

\end{lstlisting}

\subsection{Void Type}
\begin{table}[htbp]
\centering
\begin{tabular}{|c|c|}
\hline
Void Type & Generated Void \\ \hline
( void)   & void           \\ \hline
\end{tabular}
\caption[voidTypeMapping]{The table describes the generated C++ code for a void type.}
\label{tab:voidTypeMap}
\end{table}
The void type is used in most cases inside a Func Type to convey the absence of either input or output parameters. The void type is mapped to a simple `void' in C++. Table \ref{tab:voidTypeMap} shows the mapping
\subsection{Tuple Type}
Tuple types are used to define data structures which can have data of different types. While generating C++ code, the tuple types are used to generate structs that are in turn used to support data structures containing heteregenous data.
\subsection{Func Type}
Func types are associated with function definitions and function handles. They contain information about the types of the input and out parameters of the function. The function types are used for generating function definition. More information about how function definitions are generated in given in section \ref{sec:functions}.
\section{Modules}
Modules are top-level constructs in VRIR. They contain one or more functions which have to be compiled to C++. A module also has an attribute called indexing which defines the indexing scheme. The indexing scheme can either be 0 or 1 indexing. More details about how the indexing attribute is used can be found in section \ref{sec:indexing}.
\section{Functions} 
\label{sec:functions}
\begin{table}[htbp]
\begin{tabular}{|l|l|}
\hline
\multicolumn{1}{|c|}{Func Type}                                                                                                                                                                                                                                       & \multicolumn{1}{c|}{Generated Function}                                                          \\ \hline
\begin{tabular}[c]{@{}l@{}} (function babai \\
   (functype \\
   (intypes \\
   ( arraytype :layout colmajor :ndims 2(float64 :ctype 0)) \\
	( arraytype :layout colmajor :ndims 2(float64 :ctype 0))) \\
   (outtypes \\
   ( arraytype :layout colmajor :ndims 2(float64 :ctype 0)))) \\
	(arglist \\
   (arg :id0 \\
   )(arg :id1 \\
   )) \\
	(body ... )
\end{tabular} & \begin{tabular}[c]{@{}l@{}}VrArrayPtrF64 \\ babai (VrArrayPtrF64 R,VrArrayPtrF64 y)\end{tabular} \\ \hline
\end{tabular}
\caption[funcTypeMap]{The table shows an example func type and the equivalent function signature that was generated using the func type.}
\label{tab:funcTypeMap}
\end{table}
Functions in VRIR are compiled to separate functions in C++. The function node in VRIR has multiple children all of which are required to generate the C++ code for the function. The list of children node of the function node and their role in the code generation process is as follows :
\begin{itemize}
\item Name : The function name represents the name of the function. A C++ function of the same name is generated.   
\item Arglist : The arglist is a list of integers which are the Ids of the input arguments in the symbol table.
\item Func type : The Func type is used for generating the input argument types and the return type.
\item Body : The body represents the body of the function. It consists of a list of statements. 
\end{itemize}
The table \ref{tab:funcTypeMap} depicts how a function node in VRIR is converted to a C++ function.
\subsection{Return types in VRIR}
\begin{lstlisting}[language=c,caption={Generated structure to handle multiple returns.},label={lst:multReturn}]
//Structure definition
typedef struct struct_adapt_ret { 
   VrArrayPtrF64 ret_data0;
   double ret_data1;
   double ret_data2;
   
   struct_adapt_ret(VrArrayPtrF64 ret_data0,double ret_data1,double ret_data2) :ret_data0(ret_data0),ret_data1(ret_data1),ret_data2(ret_data2)
   {
   }
   
}struct_adapt_ret;

//Function declaration
struct_adapt_ret adapt(double a,double b,double sz_guess,double tol);
\end{lstlisting}
C++ only permits single return types. On the other hand, VRIR supports multiple return types. In order to bridge this difference in semantics, we generate a struct definition whose fields are of the same types as the return types. A parameterized constructor is also provided to assign the different variables that need to be returned to the member fields of the struct. The structure definition and the function definition that returns the structure is shown in listing \ref{lst:multReturn}. The struct name is generated using the name of the function. The format of a struct for multiple returns is, struct\_<function name>\_ret. This allows the calling function to determine the name of the struct while declaring a variable. 
\section{Operators}
\label{sec:operators}
			\begin{table}[htbp]
					\centering
					\begin{tabular}{|c|c|}
					\hline
					VRIR operators & C++ Operators \\ \hline
					plus           & +             \\ \hline
					minus          & -             \\ \hline
					mult           & *             \\ \hline
					div            & /             \\ \hline
					and            & \&\&           \\ \hline
					or             & ||            \\ \hline
					lt             & \textless     \\ \hline
					leq            & \textless=    \\ \hline
					gt             & \textgreater  \\ \hline
					geq            & \textgreater= \\ \hline
					eq             & ==            \\ \hline
					neq            & !=            \\ \hline
					\end{tabular}
					\caption[opMap]{VRIR operators to C++ operators Mapping. The table shows the C++ operators to which the VRIR operators are mapped.}
					\label{tab:opMap}
					\end{table}
					\begin{table}[htbp]
					\centering
					\begin{tabular}{|c|c|c|c|}
					\hline
					VRIR Lib Call                               & Operand 1 & Operand 2 & C++ function \\ \hline
					\multirow{2}{*}{Matrix Multiplication}      & Array     & Array     & mmult        \\ \cline{2-4} 
					& Array     & Scalar    & scal\_mult   \\ \hline
					\multirow{2}{*}{Elementwise Multiplication} & Array     & Array     & vec\_mult    \\ \cline{2-4} 
					& Array     & Scalar    & scal\_mult   \\ \hline
					\multirow{2}{*}{Matrix Left Division}       & Array     & Array     & mat\_ldiv    \\ \cline{2-4} 
					& Array     & Scalar    & scal\_div    \\ \hline
					\multirow{2}{*}{Matrix Right Division}      & Array     & Array     & mat\_rdiv    \\ \cline{2-4} 
					& Array     & Scalar    & scal\_div    \\ \hline
					\multirow{2}{*}{Elementwise Division}       & Array     & Array     & elem\_div    \\ \cline{2-4} 
					& Array     & Scalar    & scal\_div    \\ \hline
					\multirow{2}{*}{Array Addition}             & Array     & Array     & vec\_add     \\ \cline{2-4} 
					& Array     & Scalar    & scal\_add    \\ \hline
					\multirow{2}{*}{Array Subtraction}          & Array     & Array     & vec\_sub     \\ \cline{2-4} 
					& Array     & Scalar    & scal\_minus  \\ \hline
					Array Copy                                  & Array     & Array     & vec\_copy    \\ \hline
					Matrix Transpose                            & Array     & -         & transpose    \\ \hline
					\end{tabular}
					\caption[List of operations on Arrays]{The table shows the different C++ functions array operators are mapped to. }
					\label{tab:arrayOpMap}
					\end{table}
Arithemetic operators in VRIR, such as plus, minus, mult, div can only have scalar operands. Hence generating C++ code these operators in straightforward. They are mapped to the operators internally supported by C++. Thus plus is mapped to the `+' operator in C++, minus is mapped to `-'. The complete list is given in table \ref{tab:opMap}. \\
					Operations on arrays, on the other hand, are mapped to the LibCall expression in VRIR. Since C++ does not have internal operators for arrays we implemented function to support these operations on arrays. These functions are housed inside the language-specific runtime library. Where ever possible these functions made calls to BLAS functions for enchanced performance. In case of the \matlab\cite{matlab} runtime, we use the Intel Math Kernel Library\cite{mkl} or MKL implementation of BLAS and in case of Python\cite{python}, we use the openBLAS\cite{openblas} implementation.
\section{Statements}
VRIR supports various statements such as assignments, for-loops, while-loops etc. Most of the statements are directly supported in C++. The assignment, for and return statements have special cases which need to be supported. VRIR also supports the parallel for statement. This statement is compiled to an equivalent for loop with OpenMP pragmas in C++.
\subsection{Assignment Statement}
While generating C++ code for the assignment statement, we had to take into account different variations of the statement as well generate additional code. Different variations include statements with an array slice operation on the left hand side, statements containing function calls on the right hand side which have multiple returns etc. 
\subsubsection{Simple Assignment Statements}
Simple assignment statements are used when the left hand side is a name expression or an index expression without the array slice operator. The number of expressions on the left hand side can not be more than one. An example of a simple assignment statement is given in table \ref{tab:simpleAssignment}.
	\begin{table}[htbp]
	\begin{tabular}{|l|l|}
					\hline
							VRIR & Generated C++ code\\
							\hline
{
\begin{lstlisting}[frame=none, language=lisp, label={lst:SimpleAssignment},numbers=none]
(assignstmt
 (lhs
	(name :id 5
	 (float64 :ctype 0)
	)
 )
 (rhs
	(index :arrayid 0 :copyslice %0
	 (arraytype :layout colmajor :ndims 2
				(float64 :ctype 0)
	 )
	 (indices
				(index :boundscheck %1 :negative %0
					 (name :id 4
							(int64 :ctype 0)
					 )
				)
	 )
	)
 )
)
\end{lstlisting}
} & 
{
\begin{lstlisting}[frame=none, language=c, label={lst:SimpleAssignment},numbers=none]
temp = VR_GET_DATA_F64(A)[(i - 1)];
\end{lstlisting}
} \\
\hline
\end{tabular}
\caption[Simple Assignment Statement]{ The table shows an example of the simple assignment statement in VRIR and the equivalent C++ code that is generated from it.}
\label{tab:simpleAssignment}
\end{table}
\subsubsection{Assignment Statements with Array Slice set}
\begin{table}[htbp]
\begin{tabular}{|c|c|}
\hline
VRIR & C++ backend \\
\hline
{
\begin{lstlisting}[language=lisp, frame=none, numbers=none]
(assignstmt
	(lhs
		(index :arrayid 12 :copyslice %0 
			( arraytype :layout colmajor :ndims 2
			 	(float64 :ctype 0))
		  (indices
				(index :boundscheck %1 :negative %0
			 		(range :exclude %0
			  		(start 
			   			(plus
								(int64 :ctype 0)
								(lhs
				 					(name :id 13
				  					(int64 :ctype 0)
									)
								)
								(rhs
				 					(realconst :ival 1
				  					(int64 :ctype 0)))))
			  		(stop 
			   			(name :id 8
								(float64 :ctype 0)
			   			)
			  		)
			 	)
			)
		 )
	 )
	)
	(rhs
		<RHS Expression>)
	)
\end{lstlisting}
} & 
{
\begin{lstlisting}[language=c,frame=none, numbers=none]
rrk.setArraySliceSpec
	(<RHS Expression>, 
	VrIndex((k + 1),n,1));
		\end{lstlisting}
} \\
\hline
\end{tabular}
\caption[Assignment with array slice set]{Table shows VRIR with array slicing on the LHS and the equivalent C++ code that is generated.}
\label{tab:sliceAssign}
\end{table}
In assignment statements which fall under this category, the left hand side expression is an index expression with atleast one slice index. The right hand side can be any expression. The array slice set operation allows a region of the array to be assigned values. Since C++ does not support this, we have provide a function in the runtime library which implements this. During code generation, the assignment statement is compiled to the function callimplementing array slice set. The parameters to this call are the array variable of index expression on the left hand side, the right hand side expression and the set of indices which define the region of the array to which the values have to be assigned to. The indices are converted to VrIndex structs. More information to VrIndex can be found sub-section \ref{subsec:vrindex}. Table \ref{tab:sliceAssign} shows a VRIR representation with a slice operation on the left hand side and the equivalent C++ code that is generated.
\subsubsection{Assignment statements that can be optimised  for redundant memory allocations}
\begin{table}[htbp]
\begin{tabular}{|c|c|}
\hline 
VRIR & C++ backend \\
\hline 
{
\begin{lstlisting}[language=lisp, frame=none, numbers=none]
(assignstmt
	(lhs
		(name :id 5
   			( arraytype :layout colmajor :ndims 2
				(float64 :ctype 0)
			)
		)
	)
	(rhs
		(libcall :libfunc mmult
			( arraytype :layout colmajor :ndims 2
				(float64 :ctype 0))
			(args
   				(name :id 5
   					( arraytype :layout colmajor :ndims 2
						(float64 :ctype 0)
					)
				)
				(name :id 5
					( arraytype :layout colmajor :ndims 2
						(float64 :ctype 0)
					)
				)
			)
		)
	)
)
\end{lstlisting}
} & 
{
\begin{lstlisting}[language=c,frame=none, numbers=none]
BlasDouble::mmult
	(CblasColMajor,CblasNoTrans
	,CblasNoTrans
	,B,B, &B);
\end{lstlisting}
} \\
\hline
\end{tabular}
\caption[Assignment with Memory optimisation]{Table shows VRIR with array operations on the LHS and the equivalent C++ code that is generated and optimised.}
\label{tab:memAssignment}
\end{table}
One of the contributions of the thesis was an optimisation where redundant memory allocations during operations on arrays are removed. This optimisation is implemented by passing the array to which the result is assigned to, as a parameter to the function implementing the array operation. This array is the left hand side expression of the assignment statement whereas the right hand side expression is the function call. More information on the optimisation can be found in section \ref{sec:memoptimise}. Table \ref{tab:memAssignment} gives an example of a VRIR representation which can potentially be optimised and the generated C++ code. 

\subsubsection{Assignment Statements with multiple LHS expressions}
\begin{table}[htbp]
\begin{tabular}{|c|c|}
\hline 
VRIR & C++ backend \\
\hline 
{
\begin{lstlisting}[language=lisp, frame=none, numbers=none]
(assignstmt
  (lhs
    (tuple
      (tupletype
        (float64 :ctype 0)
        ( arraytype :layout colmajor:ndims 2
          (float64 :ctype 0)
        )
        ( arraytype :layout colmajor:ndims 2
          (float64 :ctype 0))
        ( arraytype :layout colmajor:ndims 2
          (float64 :ctype 0)
        )
      )
      (elems
        (name :id 3
          (float64 :ctype 0)
        )
        (name :id 8
          ( arraytype :layout colmajor:ndims 2
            (float64 :ctype 0)
          )
        )
        (name :id 4
          ( arraytype :layout colmajor:ndims 2
            (float64 :ctype 0)))  
        (name :id 9
          ( arraytype :layout colmajor:ndims 2
            (float64 :ctype 0)
          )
        )
      )
    )
  )
  (rhs
      <RHS Expression>
  )
)
\end{lstlisting}
} & 
{
\begin{lstlisting}[language=c,frame=none, numbers=none]
struct_spqr_ret var_spqr1 = <rhsExpr> 
nr = var_spqr1.ret_data0;
S = var_spqr1.ret_data1;
rx = var_spqr1.ret_data2;
rn = var_spqr1.ret_data3;
\end{lstlisting}
} \\
\hline
\end{tabular}
\caption[Assignment with multiple LHS expressions]{Table shows VRIR with multiple expressions on the LHS and the equivalent C++ code that is generated.}
\label{tab:multAssignment}
\end{table}
Assignment statements can have multiple expressions on the LHS when the RHS expression is a function call with multiple returns. As mentioned in section \ref{sec:functions}, functions with multiple returns are handled by returning a struct containing the return values. In the assignment statement the expression on the LHS are replaced by a struct variable and additional code is generated to assign the values in the struct to the LHS expressions. Table \ref{tab:multAssignment} shows the C++ code that is generated from VRIR.

\subsection{For Statement}
\begin{table}[htbp]
\begin{tabular}{|c|c|}
\hline 
VRIR & C++ backend \\
\hline 
{
\begin{lstlisting}[language=lisp, frame=none, numbers=none]
(forstmt
	(itervars
		(sym :id 7)
	)
	(loopdomain
		( domain 
			( domaintype :ndims 1 
				(int64 :ctype 0)
			)
			(range :exclude %0
				(start
					(realconst :ival 1
						(int64 :ctype 0)
					)
				)
				(stop
					(name :id 3
						(float64 :ctype 0)
					)
				)
			)
		)
	)
	(body
		<Loop Body>
	)
)

\end{lstlisting}
} & 
{
\begin{lstlisting}[language=c,frame=none, numbers=none]
for(h=1;h<= k;h=h+static_cast<long>(1)) {
	<Loop Body>
}
\end{lstlisting}
} \\
\hline
\end{tabular}
\caption[For Statement]{The table shows a for statement node in VRIR and its equivalent C++ code}
\label{tab:forStmt}
\end{table}
The For statement node in VRIR is compiled to a for loop is C++. The domain node of the for statement is used to determine the ranges over which the for loop iterates. If there are multiple ranges in the domain node, the for statement node is compiled into multiple nested loops. The names of the loop variables is determined by fetching their IDs from the itervar node and using their IDs to look up their names in the symbol table. An example of the generated in given in table \ref{tab:forStmt}
\begin{table}[htbp]
\begin{tabular}{|l|l|}
\hline

For loop excluding stop value & 
{
\begin{lstlisting}[language=c,frame=none, numbers=none]
for(h=1; h< k; h=h+static_cast<long>(1)) {
	<Loop Body>
}
\end{lstlisting}
}
 \\
\hline 

For loop including stop value & 
{
\begin{lstlisting}[language=c,frame=none, numbers=none]
for(h=1; h<= k; h=h+static_cast<long>(1)) {
	<Loop Body>
}
\end{lstlisting}
} \\
\hline
\end{tabular}
\caption[Use of exclude flag in For statement]{Table shows a C++ for loop with the exclude flag set to 0 and 1.}
\label{tab:excludeFor}
\end{table}
\subsubsection{Determining inclusion of the Stop expression}
The loop domain node of the for statement gives the start, stop and step expressions for each range. These expressions are used to generate the initialisation, condition and increment statements of the C++ for loop. We have to determine whether the range is  inclusive of the stop value.  This is done using the exclude flag of the range expression. If the flag is set to `\%1', the value is excluded whereas it is included if the flag is set to `\%0'. In case of an excluded stop value, the `<' operator is used in the condition statement and the `<=' operator is used in case of an included stop value. Table \ref{tab:excludeFor} gives an example the generated for loops with and without including the stop value.
\subsubsection{Determining loop direction}
\subsection{Return Statement}
\subsection{Break Statement}
\subsection{If Statement}
\subsection{Continue Statement}
\subsection{While Statement}
\subsection{Parallel For Statement}
\subsection{Statement List}
\section{Expressions}
\section{Basic Indexing}
\label{sec:indexing}
\subsection{ Generating a single index value from multiple indices}
\subsection{Negative Indexing}
\section{Advanced Indexing}
\subsection{VrIndex}
\label{subsec:vrindex}
\subsection{Array Slicing}
\section{Symbol Table}
\section{Run time library}
