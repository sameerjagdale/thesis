An important contribution of the thesis is the static generation of C++ code from VRIR. As described in \chapref{chap: Background}, VRIR is a high level strongly typed AST designed to support easy compilation of a wide range of Array based languages and hence supports different indexing schemes such as 0-indexing and 1-indexing and negative indexing and different array layout schemes like row-major and column-major  among others. C++ on the other hand does not have an in-built support for arrays, and only supports 0-indexing and a row major layout. This chapter describes in detail the challenges of compiling from VRIR to C++ and  how different nodes in VRIR are mapped to C++ constructs including those which C++ does not implicitly support.
%\section{Advantages of partial compilation over complete compilation}
\section{Challenges}
This section lists the various challenges in compiling from VRIR to C++.
\subsection{Arrays}
\subsection{Multiple returns}
\section{Mapping types}
\section{Operators}
\section{VrArrays}
\section{Statements}
\subsection{Assignment statement}
\subsection{For Statement}
\section{Basic Indexing}
\subsection{ Generating single index value from multiple indices}
\subsection{negative indexing}
\section{Advanced indexing}
\subsection{VrIndex}
\subsection{Array Slicing}

